%% Az itrodalomjegyzek keszitheto a BibTeX segedprogrammal:
%\bibliography{diploma}
%\bibliographystyle{plain}

%VAGY "k�zzel" a k�vetkez� m�don:
{\footnotesize{
\begin{thebibliography}{9}
%10-n�l kevesebb hivatkoz�s eset�n
\addcontentsline{toc}{section}{Irodalomjegyz�k}
%\begin{thebibliography}{99}
% 10-n�l t�bb hivatkoz�s eset�n

%Elso szerzok vezetekneve alapjan �b�c�rendben rendezve.
{\setstretch{1.0}
\begin{comment}
%foly�irat cikk: szerzok(k), a foly�irat neve kiemelve,
%az evfolyam felkoveren, zarojelben az evszam, vegul az oldalszamok es pont.
\bibitem{Gischer}
J. L. Gischer,
The equational theory of pomsets.
\emph{Theoret. Comput. Sci.}, \textbf{61}(1988), 199--224.

%k�nyv (szerzo(k), a k�nyv neve kiemelve, utana a kiado, a kiado szekhelye, az evszam es pont.)
\bibitem{Pin}
J.-E. Pin,
\emph{Varieties of Formal Languages},
Plenum Publishing Corp., New York, 1986.
\end{comment}
\bibitem{SegmOfRibs}
Jaesung Lee and Anthony P. Reeves,
\emph{Segmentation of Individual Ribs from Low-dose Chest CT},
School of Electrical and Computer Engineering
Cornell University, Ithaca, NY, USA, 2010

}


\end{thebibliography}

\todoi{10-es bet�m�ret kell}

}
}
